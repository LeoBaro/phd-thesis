\chapter*{Academic activities and publications}
\addcontentsline{toc}{chapter}{Academic activities and publications}

\textbf{Conferences and meetings}:
\begin{itemize}
    \item CTA meeting at Legnaro, 29/11/2018, spoken presentation on \textit{rta-lib}, a software library to stream events data to the system that performs real-time analysis.
    \item CTA meeting at Montpellier (FR), 31/03/2019, spoken presentation on a prototype for the science alert generation system of CTA.
    \item INAF meeting at Pula, 16/09/2019, spoken presentation on features selection techniques in machine learning.
    \item INAF remote meeting AGILE, 26/02/2020, spoken presentation of \textit{Agilepy}, a python library to analyze the AGILE space telescope data.
\end{itemize}

\textbf{R\&D activities}:
\begin{itemize}
    \item Lead developer of Agilepy, an open-source Python package developed at INAF/OAS Bologna to analyze the \textit{GRID} instrument of the AGILE space telescope \url{https://github.com/AGILESCIENCE/Agilepy}.
    \item Development of the SAG-DQ software package for the science alert generation system of CTA. Two python libraries have been developed to meet the data quality analysis requirements. The first library, called \textit{rta-dq-lib}, implements the core logic of data quality analysis, while the second, called \textit{rta-dq-pipe}, manages the parallel execution of multiple data quality analysis pipelines.
    \item Development of the SAG-SUP software package for the Science Alert Generation system of CTA. It's implemented in Java and based on the Alma Common Software (ACS) that allows the development of distributed systems. It supervises the analysis pipelines of the Science Alert Generation system, and it interfaces with the rest of the Array Control and Data Acquisition system of CTA.
    \item Member of the integration team of the Array Control and Data Acquisition system of CTA: weekly remote meetings to discuss requirements, use cases and integration tests between the subsystems. 
    \item Ph.D. thesis on the investigation of deep learning-based anomaly detection for real-time gamma-ray analysis to detect gamma-ray bursts. Source code available at: \url{https://github.com/LeoBaro/Real-Time-Anomaly-Detection-of-Gamma-Ray-Bursts-for-the-Cherenkov-Telescope-Array}.
\end{itemize}


\textbf{Missions}:
\begin{itemize}
    \item Shift as operator for the LST-1 telescope located at La Palma (ES) from 25/07/2021 to 19/08/2021.
    \item Exchange visit collaboration at the Laboratoire d'Annecy de Physique des Particules in Annecy (FR) to work on integrating the SAG-SUP with the SAG-RECO software packages in the context of the Array Control and Data Acquisition system of CTA. From 29/08/2022 to 07/11/2022.
\end{itemize}

\textbf{Publications}:
\begin{itemize}
    \item Baroncelli, L., “Rta-dq-lib: Software Library to Perform Online Data Quality Analysis of Scientific Data”, in <i>Astronomical Society of the Pacific Conference Series</i>, 2022, vol. 532, p. 365. doi:10.48550/arXiv.2105.08648.
    \item A publication based on this work is in the making. 
\end{itemize}

\textbf{Co-authored publications}:
\begin{itemize}
\item Ursi, A., “GRB 230104A: AGILE/MCAL detection of a burst”, GCN.33148, 2023.
\item Ursi, A., “GRB 230102A: AGILE/MCAL detection”, GCN.33138, 2023.
\item Di Piano, A., “AGILE detection of gamma-ray activity from blazar 4C29.45”, ATel15853, 2023.
\item Panebianco, G., “GRB 221226A: AGILE detection”, GCN.33113, 2022.
\item Casentini, C., “GRB 221221A: AGILE detection”, GCN.33093, 2022.
\item Verrecchia, F., “AGILE detection of increasing gamma-ray activity of blazar PKS 1424-418”, ATel15818, 2022.
\item Casentini, C., “GRB 221126A: AGILE detection”, GCN.32992, 2022.
\item Bulgarelli, A., “Ongoing gamma-ray flare from the PKS 0805-07 detected by AGILE”, ATel15768, 2022.
\item Abe, S., “Multi-wavelength study of the galactic PeVatron candidate LHAASO J2108+5157”, <i>arXiv e-prints</i>, 2022. doi:10.48550/arXiv.2210.00775.
\item Piano, G., “GRB 221009A (Swift J1913.1+1946): AGILE/GRID detection”, GCN.32657, 2022.
\item Ursi, A., “GRB 221009A (Swift J1913.1+1946): AGILE/MCAL detection”, GCN.32650, 2022.
\item Piano, G., “GRB 221009A/Swift J1913.1+1946: AGILE-GRID detection”, ATel15662, 2022.
\item Piano, G., “AGILE detection of gamma-ray activity from the FSRQ PKS 1830-211”, ATel15649, 2022.
\item Parmiggiani, N., “The online observation quality system software architecture for the ASTRI Mini-Array project”, in <i>Society of Photo-Optical Instrumentation Engineers (SPIE) Conference Series</i>, 2022, vol. 12189. doi:10.1117/12.2629278.
\item Addis, A., “The Gamma-Flash real-time data pipeline for ground observation of terrestrial gamma-ray flashes”, in <i>Society of Photo-Optical Instrumentation Engineers (SPIE) Conference Series</i>, 2022, vol. 12189. doi:10.1117/12.2627961.
\item Conforti, V., “The Array Data Acquisition System software architecture of the ASTRI Mini-Array project”, in <i>Society of Photo-Optical Instrumentation Engineers (SPIE) Conference Series</i>, 2022, vol. 12189. doi:10.1117/12.2626600.
\item Bulgarelli, A., “The Software Architecture and development approach for the ASTRI Mini-Array gamma-ray air-Cherenkov experiment at the Observatorio del Teide”, in <i>Society of Photo-Optical Instrumentation Engineers (SPIE) Conference Series</i>, 2022, vol. 12189. doi:10.1117/12.2629164.
\item Verrecchia, F., “Gamma-ray rebrightening of the blazar PKS 1424-418 detected by AGILE”, ATel15533, 2022.
\item Piano, G., “Ongoing gamma-ray flare from the blazar PKS 1424-418 detected by AGILE”, ATel15527, 2022.
\item Bulgarelli, A., “Agilepy: A Python Framework for AGILE Data Analysis”, in <i>Astronomical Society of the Pacific Conference Series</i>, 2022, vol. 532, p. 505.
\item Conforti, V., “Performance Improvement of the Data Acquisition System to Support the Observation Quality System of the ASTRI Mini-Array”, in <i>Astronomical Society of the Pacific Conference Series</i>, 2022, vol. 532, p. 393.
\item Parmiggiani, N., “RTApipe, a Framework to Develop Astronomical Pipelines for the Real-time Analysis of Scientific Data”, in <i>Astronomical Society of the Pacific Conference Series</i>, 2022, vol. 532, p. 139. doi:10.48550/arXiv.2105.08611.
\item Parmiggiani, N., “The RTApipe framework for the gamma-ray real-time analysis software development”, <i>Astronomy and Computing</i>, vol. 39, 2022. doi:10.1016/j.ascom.2022.100570.
\item Bulgarelli, A., “The Science Alert Generation system of the Cherenkov Telescope Array Observatory.”, in <i>37th International Cosmic Ray Conference</i>, 2022. doi:10.22323/1.395.0937.
\item Parmiggiani, N., “The AGILE real-time analysis pipelines in the multi-messenger era”, in <i>37th International Cosmic Ray Conference</i>, 2022. doi:10.22323/1.395.0933.
\item CTA-LST Project, T., “Development of an advanced SiPM camera for the Large Size Telescope of the Cherenkov TelescopeArray Observatory”, in <i>37th International Cosmic Ray Conference</i>, 2022. doi:10.22323/1.395.0889.
\item CTA-LST Project, T., “Status and results of the prototype LST of CTA”, in <i>37th International Cosmic Ray Conference</i>, 2022. doi:10.22323/1.395.0872.
\item CTA-LST Project, T., “First follow-up of transient events with the CTA Large Size Telescope prototype”, in <i>37th International Cosmic Ray Conference</i>, 2022. doi:10.22323/1.395.0838.
\item CTA-LST Project, T., “Physics Performance of the Large Size Telescope prototype of the Cherenkov Telescope Array”, in <i>37th International Cosmic Ray Conference</i>, 2022. doi:10.22323/1.395.0806.
\item Lopez-Coto, R., “Deep-learning-driven event reconstruction applied to simulated data from a single Large-Sized Telescope of CTA”, in <i>37th International Cosmic Ray Conference</i>, 2022. doi:10.22323/1.395.0771.
\item Ohtani, Y., “Cross-calibration and combined analysis of the CTA-LST prototype and the MAGIC telescopes”, in <i>37th International Cosmic Ray Conference</i>, 2022. doi:10.22323/1.395.0724.
\item Kobayashi, Y., “Camera Calibration of the CTA-LST prototype”, in <i>37th International Cosmic Ray Conference</i>, 2022. doi:10.22323/1.395.0720.
\item Blanch, O., “Commissioning of the camera of the first Large Size Telescope of the Cherenkov Telescope Array”, in <i>37th International Cosmic Ray Conference</i>, 2022. doi:10.22323/1.395.0718.
\item Alispach, C. M., “Reconstruction of extensive air shower images of the Large Size Telescope prototype of CTA using a novel likelihood technique”, in <i>37th International Cosmic Ray Conference</i>, 2022. doi:10.22323/1.395.0716.
\item Foffano, L., “Monitoring the pointing of the Large Size Telescope prototype using star reconstruction in the Cherenkov camera”, in <i>37th International Cosmic Ray Conference</i>, 2022. doi:10.22323/1.395.0712.
\item CTA-LST Project, T., “Analysis of the Cherenkov Telescope Array first Large Size Telescope real data using convolutional neural networks”, in <i>37th International Cosmic Ray Conference</i>, 2022. doi:10.22323/1.395.0703.
\item Di Piano, A., “Detection methods for the Cherenkov Telescope Array at very-short exposure times”, in <i>37th International Cosmic Ray Conference</i>, 2022. doi:10.22323/1.395.0694.
\item Parmiggiani, N., “The Online Observation Quality System for the ASTRI Mini Array.”, in <i>37th International Cosmic Ray Conference</i>, 2022. doi:10.22323/1.395.0692.
\item Zanin, R., “CTA – the World's largest ground-based gamma-ray observatory”, in <i>37th International Cosmic Ray Conference</i>, 2022. doi:10.22323/1.395.0005.
\item Ursi, A., “The Second AGILE MCAL Gamma-Ray Burst Catalog: 13 yr of Observations”, <i>The Astrophysical Journal</i>, vol. 925, no. 2, 2022. doi:10.3847/1538-4357/ac3df7.
\item Panebianco, G., “AGILE detection of renewed Gamma-ray activity from 3C 454.3.”, ATel15782, 2022.
\item Pittori, C., “AGILE detects a gamma-ray rebrightening of the blazar PKS 0903-57”, ATel15086, 2021.
\item Piano, G., “AGILE detection of intense transient gamma-ray activity from Cygnus X-3 emerging from a prolonged quenching state”, ATel15009, 2021.
\item Pittori, C., “AGILE detection of gamma-ray flaring activity from the FSRQ TXS 0646-176”, ATel14981, 2021.
\item Piano, G., “AGILE detection of enhanced gamma-ray activity from the blazar BL Lac”, ATel14839, 2021.
\item Verrecchia, F., “AGILE Observations of Fast Radio Bursts”, <i>The Astrophysical Journal</i>, vol. 915, no. 2, 2021. doi:10.3847/1538-4357/abfda7.
\item Verrecchia, F., “AGILE detection of gamma-ray flaring activity from the FSRQ Ton 0599”, ATel14785, 2021.
\item Piano, G., “AGILE confirmation of the gamma-ray flaring activity from the blazar BL Lac”, ATel14782, 2021.
\item Piano, G., “AGILE detection of transient gamma-ray activity from Cygnus X-3 during a prolonged quenched state”, ATel14780, 2021.
\item Bošnjak, Ž., “Multi-messenger and transient astrophysics with the Cherenkov Telescope Array”, <i>arXiv e-prints</i>, 2021. doi:10.48550/arXiv.2106.03621.
\item Bulgarelli, A., “Agilepy: A Python framework for scientific analysis of AGILE data”, <i>arXiv e-prints</i>, 2021. doi:10.48550/arXiv.2105.08474.
\item Piano, G., “Transient gamma-ray emission from Cygnus X-3 detected by AGILE during the current quenched/hypersoft state”, ATel14662, 2021.
\item Bulgarelli, A., “AGILE confirmation of the gamma-ray flaring activity from the two blazars PKS 0514-459 and TXS 1700+685”, ATel14634, 2021.
\item Tavani, M., “An X-ray burst from a magnetar enlightening the mechanism of fast radio bursts”, <i>Nature Astronomy</i>, vol. 5, pp. 401–407, 2021. doi:10.1038/s41550-020-01276-x.
\item Abdalla, H., “Sensitivity of the Cherenkov Telescope Array for probing cosmology and fundamental physics with gamma-ray propagation”, <i>Journal of Cosmology and Astroparticle Physics</i>, vol. 2021, no. 2, 2021. doi:10.1088/1475-7516/2021/02/048.
\item Tavani, M., “Gamma-Ray and X-Ray Observations of the Periodic-repeater FRB 180916 during Active Phases”, <i>The Astrophysical Journal</i>, vol. 893, no. 2, 2020. doi:10.3847/2041-8213/ab86b1.
\item Lucarelli, F., “AGILE detection of enhanced gamma-ray activity from the blazar PKS 0903-57”, ATel13602, 2020.
\item Bulgarelli, A., “Second AGILE catalogue of gamma-ray sources”, <i>Astronomy and Astrophysics</i>, vol. 627, 2019. doi:10.1051/0004-6361/201834143.
\item Bulgarelli, A., “VizieR Online Data Catalog: Second AGILE catalogue of gamma-ray sources (Bulgarelli+, 2019)”, <i>VizieR Online Data Catalog</i>, 2019.
\item Verrecchia, F., “AGILE Observations of the Gravitational-wave Source GW170817: Constraining Gamma-Ray Emission from an NS-NS Coalescence”, <i>The Astrophysical Journal</i>, vol. 850, no. 2, 2017. doi:10.3847/2041-8213/aa965d.


  
\end{itemize}
