\section{Anomaly detection method}
\label{c:conclusions}
This method overcomes the limitations of the analysis techniques described in \autoref{ss:ffov-ml} and \autoref{ss:li-ma}. No assumption on the background model needs to be assumed: the semi-unsupervised learning approach enables the automatic modeling of the background since only background data is used to train the autoencoder. In addition, no hypothesis on the source model needs to be assumed: the anomaly detection approach works by detecting which samples deviate significantly from the normal data. Since there's no need to model the sources, we can overcome the limitation of unrepresentative training data sets. This technique can be applied in a streaming context, identifying outliers in real time as soon as new data points are available. This means that this method can work with overlapping sub-sequences, drastically reducing the time to perform inference. In contrast, the Li\&Ma technique needs to wait for further data since it is limited by the assumption of the statistical independence of the time bins under analysis. Furthermore, using an integration time equal to 1 second, the proposed technique can achieve faster inference, pushing the boundaries of short-term analysis. This is possible because there's no limitation on the minimum amount of signal the method needs to process, overcoming the second limitation of the Li\&Ma technique. Performing real-time analysis of each second of data is also possible because the computational cost of the method is very low due to the autoencoder's simple architecture and the data's size. In addition, multiple models can run in parallel, enabling multi-timescales analysis. According to the expected properties of the signals to detect, the integration time value and the length of the sub-sequences can be chosen accordingly. 


\section{Outlook and future work}
Give an outlook regarding your expectations for the overall development of your chosen field of research. List topics that were not covered by your work. Identify and discuss potential avenues for follow-up research that you consider worthwhile pursuing.