The main goal of this thesis work is to address the problem of source detection, described in [ref], while overcoming the limitations of the techniques that are currently being proposed for the real-time analysis of data produced by the telescopes' of the CTAO ([ref]). In addition, since this work is framed in the context of the software development for ACADA, the software requirements described in [ref] have been taken into account. This chapter is organized as follows. Section \autoref{s:Contribution-2-anomaly-detection} will describe the proposed anomaly detection technique to adress the source detection problem. Section [ref] will describe the data pipeline that has been developed to generate input data. Section [ref] will describe the p-value analysis to associate each detection with a gaussian statistical significance. Section [ref] will investigate several problems that can arise during the telescopes observations and how those problems can affect the proposed system. 

\section{Scientific use cases: serendipitous discoveries and observatrion triggered by a science-alerts}
\label{s:Contribution-2-ues-cases}
Among the astrophysical events ...
Transient sources...GRBs..
Limited in time... [TODO]
The real-time analysis of the telescopes' data allows to detect a source as soon as possible. This capability is a key feature in the context of multi-messanger and multi-wavelength astronomy because it allows to detect in real-time an astrophysical event and to communicate the finding to other observatories that, in turn, can start to observe the same sky region. This communication is in the form of a science-alert that is broadcasted into a network such as GCN [ref]. When an observatory receives a science-alert, some policies will drive the next actions depending on what the observatory was doing at that moment, on the type of the science-alert and on the observation conditions such as weather or moon phases. The method proposed in this work has been applied to two scientific use cases: serendipitous discoveries and observatrion triggered by a science-alerts. 

\subsubsection{Serendipitous discoveries}
The probability to detect a random GRB event during an observation is very low, but still possible. When the observatory is generating catalogs of known sources, it will cover large areas of the sky for HOW MUCH time. Since the field-of-view of the telescope is larger with respect to the region of interests, the probability that an unexpected event serendipitously appears increases. In this settings, the evolution of the event could be observed from the beginning. But the source must be detected as soon as possible in order to broadcast a science-alert that will maximaze the scientific return, allowing the other observatories to follow the same event from the beginning. Figure [ref] shows a sky map in which a unexprected event appears outside the region of interest but inside the telescope's field-of-view.  

\subsubsection{Follow-up observation triggered by a science-alert}
In this scenario the observatory receives a science-alert. The science-alert specify the following: the type of the event, the localization map describing the source location in terms of probabilities and ... [ref]. The observatory will interrupt the current observation and change its pointing to detect the new source in the least possible amount of time. This scenario differs from the serendipitous discovery scenario in two ways: if the localization map is much bigger than the field-of-view, multiple observation following a tiling strategy as shown in fig. [ref] are required to cover the whole localization region. Another difference is that the evolution of the event can not be observed from the beginning: there's a delay between the detection of the source that triggered the science-alert and the time taken to repoint the telescopes when the science-alert is received. The latter increases the difficult of the problem because often the most energetic and luminous part of a GRB is the promt as described in chapter [ref]. 


\subsection{Limitation of the Li\&Ma and full field-of-view maximum likelihood techniques}
\label{s:Contribution-2-Major-1-Minor-1}
As mentioned in chapter [ref], two techniques have been investigated to be used in the real-time context. The Li\&Ma technique [ref] is ..
This technique has some limitations that . 

\section{The proposed method based on anomaly detection}
\label{s:Contribution-1-anomaly-detection}
This section will describe the proposed anomaly detection technique. As explained in [ref], the goal of the anomaly detection is to spot anomalies, i.e. data samples that which deviates so much from other observations as to arouse suspicions that it was generated by a different mechanism [ref]. In this context, the data being analyzed are timeseries which points represents flux measurements. As described in [ref], the flux quantity describes how much light the source is emitting. The data pipeline to obtain the timeseries will be explained in detail in the next section [ref]. The problem of source detection has been framed inside the anomaly detection framework distinguishing between:
\begin{itemize}
	\item normal data: background only signal. 
	\item anomalous data: background and source signal.
\end{itemize}






\subsubsection{For structuring...}
Develop your idea!


%\begin{algorithm}[t]
%  \caption{How Bogosort works.}
%  \label{a:Bogosort}
%  \begin{algorithmic}[1]
%    \REQUIRE Unsorted dataset $d$, length of dataset $n$
%    \STATE shuffle $d$
%%    \FOR{$i = 2 \dots n$}
%      \IF{$d_{i-1} > d_i$}
%        \STATE goto 1
%      \ENDIF
%    \ENDFOR
%  \end{algorithmic}
%\end{algorithm}

%\begin{lstlisting}[
%caption={Classic implementation of Bogosort.},
%label=l:Bogosort,
%language=Python,
%numbers=none,
%frame=single,
%float
%]
%def is_sorted(d):
%%    for i in range(len(d) - 1):
%        if d[i] > d[i + 1]:
%            return False
%    return True

%def bogosort(d):
%    while not is_sorted(d):
%5        random.shuffle(d)
%    return data
%\end{lstlisting}


\section{Summary}
\label{s:Contribution-2-Summary}

Summarize contribution 2. Highlight what makes it relevant. Assuming that the reader knows the details of the contribution now, you should also try to clearly explain in what particular key properties this method deviates from existing research.
