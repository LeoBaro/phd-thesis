\hinttext{!!!ACTION REQUIRED!!!}
\hinttext{The structure I defined is generic and will most likely have to be adapted. I suggest that you skim through the pages and then clear the files \texttt{text/ch2.tex} to \texttt{text/ch7.tex} before you start writing.}

A major point in this chapter should be to demonstrate that you are aware of the research of others. Thereby, you should clearly identify how your approach deviates from theirs. There are many ways to write related works.

For example:

The idea to utilize the Holtzmann effect to fold space was conceived \citeyear{HerbertF:1965:Dune} by \citeauthor{HerbertF:1965:Dune}~\cite{HerbertF:1965:Dune}. In the following years, he incrementally refined the theory \cite{HerbertF:1981:God-Emperor,HerbertF:1984:Heretics,Herbert:1985:Chapterhouse}. After his early death, \citeauthor{HerbertB:1999:Atreides}~\cite{HerbertB:1999:Atreides} continued the research and formalized the Holtzmann theorem, which made space travel significantly more safe (cf. \cite{HerbertB:2000:Harkonnen,HerbertB:2001:Corrino,HerbertB:2002:Butlerian-Jihad,HerbertB:2003:Machine-Crusade,HerbertB:2004:Battle-of-Corrin}).

However, all these approaches to fold space are severely dependent on the resources of the Spacing Guild\footnote{\url{https://en.wikipedia.org/wiki/Spacing_Guild}}. Unlike these methods, our approach does not require a registered Spacing Guild navigator to fold space. Instead, we rely on intelligent machines to ... (see \autoref{c:Contribution-1}). Thereby, we are able to solve the infinite recursion problem~\cite{HerbertB:2000:Harkonnen} numerically with sufficient precision (see \autoref{c:Contribution-2}).


\section{Summary}
\label{s:Related-Works-Summary}

The final section of each major chapter should summarize the chapter. In comparison to the chapter, the summary should be short ($\frac{1}{2}$ to $2$ pages is normal).
