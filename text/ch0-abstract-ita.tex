Contesto. Il Cherenkov Telescope Array (CTA) sarà l'osservatorio terrestre di prossima generazione per lo studio dell'universo nel dominio delle altissime energie. Sarà composto da più di sessanta telescopi Cherenkov di nuova generazione che migliorano la sensibilità di un ordine di grandezza rispetto ai Imaging Atmospheric Cherenkov Telescopes (IACT) attuali, portando grandi contributi all'astrofisica delle alte energie. L'osservatorio sfrutterà il sistema software di Science Alert Generation (SAG) per l'analisi in tempo reale dei dati osservativi e per generare automaticamente allerte scientifiche. Il sistema SAG svolgerà un ruolo da protagonista nella ricerca e nel follow-up di fenomeni transienti a seguito di allerte esterne, consentendo collaborazioni multi-wavelength e multi-messenger. L'osservatorio comprenderà due array di telescopi su due siti in entrambi gli emisferi per fornire la piena copertura del cielo, massimizzando la capacità di rilevare i fenomeni più rari, come i lampi di raggi gamma (GRBs), che sono il caso scientifico di questo studio. \\
Obiettivi. Lo scopo di questo studio è indagare l'utilizzo della tecnica di anomaly detection per l'analisi in tempo reale dei dati gamma. È stata quindi sviluppata una tecnica basata sul deep learning per perseguire questo obiettivo. \\
Risultati. QUesto studio presenta una tecnica di anomaly detection basata sul deep learning per la rilevazione in tempo reale di GRBs. Le prestazioni della tecnica proposta sono valutate e confrontate con la tecnica standard di Li\&Ma,
nei due casi d'uso scientifici di \textit{serendipitous discoveries} e \textit{follow-up observations}, considerando diversi tempi di esposizione. La tecnica proposta mostra risultati promettenti e è abbastanza flessibile da consentire la ricerca di eventi transienti su più tempi scala. Non necessita di fare ipotesi sui modelli del background e della sorgente e non richiede un numero minimo di conteggi di fotoni per eseguire l'analisi, rendendola adatta per l'analisi in tempo reale. \\
Conclusioni. Miglioramenti futuri includono ulteriori test, accantonando alcune delle ipotesi semplificative assunte in questo studio, così come la correzione post-trial della significatività di rilevazione. Inoltre, dovrà essere testata la capacità di rilevare altre classi di transienti oltre ai GRBs. L'integrazione all'interno del sistema SAG e la messa in produzione nei centri di calcolo onsite, fornirebbe preziose informazioni sulle prestazioni del metodo con dati non simulati. Nel complesso, questo studio fornisce un contributo significativo al campo dell'astrofisica delle alte energie e dimostra l'efficacia della tecnica di anomaly detection per la rilevazione in tempo reale di fenomeni transienti.