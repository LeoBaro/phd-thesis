\section{Experiment Setup}
\label{s:Experiment-Setup}

This Chapter introduces the experimental measurements of the performance of the Anomaly Detection model presented in \autoref{c:Contribution-2}.
The goal is to measure the performances of the A.D. method against the Li\&Ma technique described in \autoref{sb:li-ma} to perform source detection. The sooner a detection is issued, the sooner other observatories can study the same source. For this reason, we are interested in measuring how fast the methods are to issue 5$\sigma$ detections.
In addition, we need to perform these measurements for the two scientific use cases of serendipitous discoveries and science alert reactions. In the serendipitous discoveries settings, a GRB appears in the telescope's field of view; hence the totality of the event is observed. When the observatory reacts to a science alert, its telescopes will change their pointing to observe a sky region in which the GRB event is already started. The time taken to change the pointing is considered fixed and equal to 30 seconds. Two test sets of different difficulties have been generated for this purpose.

\subsection{Dataset Generation for performance evaluations}
\label{s:Experiment-Data}
Two supervised datasets of simulated GRB signals have been generated. As stated in \autoref{c:Background}, gamma-ray transients differ among several factors: [todo]. The max flux parameter constrains the maximum luminosity of the phenomena. If the signal is too low, it will be indistinguishable from the noise. The mean of the background signal will be the reference point to tune the difficulty of the problem and it is defined by the same IRF used in the training test generation ($North\_z40\_5h\_LST$), described in \autoref{c:Contribution-2}. Figure~\ref{f:exp-max-flux-distribution-E} show that the first dataset (called $E=easy$) contains GRB simulated trials with a max flux greater or equal to the mean of the background signal. It contains 188 trials with a max flux in the interval $[1.2264e-09, \infty]$]. In the second dataset (called $H=hard$), the max flux goes down to 1$\sigma$ below the background threshold. It contains 438 trials with a max flux in the interval $[2.6785e-10, \infty]$].  

%The simulation tool can use this parameter to generate samples with different luminosity.  

\begin{figure}[t]
\centering
\includegraphics[width=1\textwidth]{figures/experiments/templates_max_flux_distributions.png}
\caption{The distribution of the maximum flux for each GRB template. The test sets E and H contain respectively 188  and 438 of GRB simulated trials, and their maximum fluxes are within: $[[1.2264e-09, \infty]]$ and $[[2.6785e-10, \infty]]$.}
\label{f:exp-max-flux-distribution-E}
\end{figure}


The simulation time is limited to 500 seconds because we are interested in the first part of the signal, given the nature of the scientific use cases described in Chapter [ref]. The trigger time that defines the GRB event's start is 250 seconds. The integration time is the same used for the training set generation for the reasons stated in \autoref{c:Contribution-2}. The autoencoder models have been trained with time series of lengths equal to 5, with each point being a flux measurement integrated in 5 seconds. Hence the same settings are used here. The stride parameter is set to 1 because this method can work with temporal bins that are statistically dependent, and this drastically reduces the time the model takes to wait for data during the online inference. Table [ref] summarizes the configuration parameters described above. A time series is labeled as anomalous if it contains at least one point of GRB signal. Figure [ref] shows some samples of GRB simulated trials.  

\subsection{P-value threshold and inference}
\label{s:P-Val-Inference}
Figures \ref{fig:ts-distribution-and-p-values-cnn-it-5}
\ref{fig:ts-distribution-and-p-values-rnn-it-5}
\ref{fig:ts-distribution-and-p-values-cnn-it-1}
\ref{fig:ts-distribution-and-p-values-rnn-it-1} show the distributions of the reconstruction errors and p-values for the A.D. CNN and RNN models and for integration times equal to 5 and 1. The following listing shows the thresholds the models use to issue detection with a $5\sigma$ significance. 
\begin{itemize}
    \item RNN model IT=5. Threshold: 0.00207 corresponding to $5.0488\sigma$.
    \item CNN model IT=5. Threshold: 0.0139 corresponding to $5.0465\sigma$.
    \item RNN model IT=1. Threshold: 0.0026 corresponding to $5.041\sigma$.
    \item CNN model IT=1. Threshold: 0.03164 corresponding to $5.0488\sigma$.
\end{itemize}

\begin{figure}
    \centering
    \subfigure{\includegraphics[width=0.5\textwidth]{figures/experiments/p_val/model_0/ts_distribution_bins_100.png}}
    
    \subfigure{\includegraphics[width=0.5\textwidth]{figures/experiments/p_val/model_0/pvalue_bins_100.png}}
    
    \caption{TS distributions and p-values for A.D. cnn implementation for integration time = 5}
    \label{fig:ts-distribution-and-p-values-cnn-it-5}
\end{figure}

\begin{figure}
    \centering
    \subfigure{\includegraphics[width=0.5\textwidth]{figures/experiments/p_val/model_1/ts_distribution_bins_100.png}}
    
    \subfigure{\includegraphics[width=0.5\textwidth]{figures/experiments/p_val/model_1/pvalue_bins_100.png}}
    
    \caption{TS distributions and p-values for A.D. rnn implementation for integration time = 5}
    \label{fig:ts-distribution-and-p-values-rnn-it-5}
\end{figure}


\begin{figure}
    \centering
    \subfigure{\includegraphics[width=0.5\textwidth]{figures/experiments/p_val/model_6/ts_distribution_bins_100.png}}
    
    \subfigure{\includegraphics[width=0.5\textwidth]{figures/experiments/p_val/model_6/pvalue_bins_100.png}}
    
    \caption{TS distributions and p-values for A.D. cnn implementation for integration time = 5}
    \label{fig:ts-distribution-and-p-values-cnn-it-5}
\end{figure}

\begin{figure}
    \centering
\subfigure{\includegraphics[width=0.5\textwidth]{figures/experiments/p_val/model_7/ts_distribution_bins_100.png}}
    \subfigure{\includegraphics[width=0.5\textwidth]{figures/experiments/p_val/model_7/pvalue_bins_100.png}}
    
    \caption{TS distributions and p-values for A.D. rnn implementation for integration time = 1}
    \label{fig:ts-distribution-and-p-values-rnn-it-1}
\end{figure}

\begin{figure}
    \centering
    \subfigure{\includegraphics[width=0.5\textwidth]{figures/experiments/p_val/model_0/ts_distribution_bins_100.png}}
    
    \subfigure{\includegraphics[width=0.5\textwidth]{figures/experiments/p_val/model_0/pvalue_bins_100.png}}
    
    \caption{TS distributions and p-values for A.D. cnn implementation for integration time = 1}
    \label{fig:ts-distribution-and-p-values-cnn-it-1}
\end{figure}

\subsection{Performance metrics}
\label{s:Peformance-Metrics}
The Anomaly Detection (CNN and RNN implementations) and Li\&Ma methods have been tested with two different datasets for different integration times. In the short-term analysis settings, the integration time is set to 5 seconds, while in the very short-term analysis settings, it is set to 1 second. The time series of lengths equal to 5 have been extracted from the original time series of each simulated trial. We are interested in computing the accuracy and false positive rate. These metrics tell us how many time series have been correctly classified and how many false positives classifications have been issued. This tells us how frequently the same GRB event is detected during its evolution. Another performance metric that we want to compute is the average time of the first $5\sigma$ detections for each GRB event and the number of detections limiting the observation duration (from $T_grb_start$ to $T_max$). Since A.D. method can infer each $IT$ seconds of data, while Li\&Ma needs to wait for at least $IT*TSL$ seconds, we expect the A.D. method to have more detections in the short-term window from the start of the GRB event. Finally, the next section will also a confusion matrix to show how many GRB events are seen by the models respectively to each other. 


\section{Results}
\label{s:Experiment-Results}

\subsection{Accuracy and False Positive Rate}
\label{s:Acc-Fpr}
These metrics tell us how many time series have been correctly classified and how many false positives classifications have been issued. This tells us how frequently the same GRB event is detected during its evolution.
Table [ref] shows these metrics for the A.D. CNN and RNN models and for different integration times, 1 and 5 seconds.

\subsection{Serendipitous discoveries use case}
\label{s:Serendipitous-Discoveries-Results}
In the scenario, the GRB event must be detected as soon as possible to broadcast a science alert to other observatories to observe the same event to enable multi-wavelength and multi-messenger analysis. 

Table~\ref{tab:Experiment-Results-E-IT-5} shows the total number of 5$\sigma$ detections performed by the two methods of A.D. and LiMa and the average time in seconds to detect a source for the first time. This time is measured starting from the trigger time that defines the start of the GRB event and not from the start of the observation.

The table shows the LiMa's robust predictions since it detected most GRBs. The A.D. method performs less robust but faster detections since it can predict every $T$ seconds. In LiMa the temporal bins must be independent and the slower prediction rate is 1  every $T*TSL$ seconds. This metric can be broken down further.


\begin{table}[ht]
\centering % used for centering table
\begin{tabular}{l c c c c} % centered columns (4 columns)
\hline\hline %inserts double horizontal lines
Model & Test Set & A.D. (CNN) & A.D. (RNN) & Li\&Ma \\ [0.5ex] % inserts table
%heading
\hline % inserts single horizontal line
Total detections & E & 128 & 143 & 161 \\ 
Accuracy & E &  68.09\% & 76.06\% & 85.64\%  \\ 
Against Li\&Ma & E &  0 & 1 & - \\ 
Mean time first detection & E &  50.74 & 47.48 & 52.02  \\ [1ex] % [1ex] adds vertical space

\hline % inserts single horizontal line
Total detections & H &  174 & 237 & 294 \\
Accuracy & H & 41.23\% & 56.16\% & 69.66\%  \\ 
Against Li\&Ma & H & 2 & 13 & - \\  
Mean time first detection & H & 52.64 s & 61.81 s & 63.18 s \\ [1ex]   

\hline %inserts single line
\end{tabular}
\caption{Result on Test Set E and Test Set H with integration time = 5: Total number of 5$\sigma$ detections performed by the two methods of A.D. and LiMa and the average time in seconds the methods took to detect a source for the first time, over 188 GRB (E) and 422 GRB (H) trials.}
\label{tab:Experiment-Results-E-IT-5} 
\end{table}



\begin{table}[ht]
\centering % used for centering table
\begin{tabular}{l c c c c} % centered columns (4 columns)
\hline\hline %inserts double horizontal lines
Model & Test Set & A.D. (CNN) & A.D. (RNN) & Li\&Ma \\ [0.5ex] % inserts table
%heading
\hline % inserts single horizontal line
Total detections & E & 128 & 143 & 161 \\ 
Accuracy & E &  68.09\% & 76.06\% & 85.64\%  \\ 
Against Li\&Ma & E &  0 & 1 & - \\ 
Mean time first detection & E &  50.74 & 47.48 & 52.02  \\ [1ex] % [1ex] adds vertical space

\hline % inserts single horizontal line
Total detections & H &  174 & 237 & 294 \\
Accuracy & H & 41.23\% & 56.16\% & 69.66\%  \\ 
Against Li\&Ma & H & 2 & 13 & - \\  
Mean time first detection & H & 52.64 s & 61.81 s & 63.18 s \\ [1ex]   

\hline %inserts single line
\end{tabular}
\caption{Result on Test Set E and Test Set H with integration time = 1 : Total number of 5$\sigma$ detections performed by the two methods of A.D. and LiMa and the average time in seconds the methods took to detect a source for the first time, over 188 GRB (E) and 422 GRB (H) trials.}
\label{tab:Experiment-Results-E-IT-1} 
\end{table}


\begin{figure}[t]
\centering
\includegraphics[width=1\textwidth]{figures/experiments/ad_vs_li_ma_first_detections_testset_e_id_1.png}
\caption{Result on Test Set E: 5$\sigma$ detection times distributions for A.D. and LiMa over 188 GRB trials.}
\label{f:ad-vs-lima-first-detection}
\end{figure}


\begin{figure}[t]
\centering
\includegraphics[width=1\textwidth]{figures/experiments/ad_vs_li_cumulative_testset_e_id_1.png}
\caption{Result on Test Set E: Total number of 5$\sigma$ detections (out of 188 GRB trials) performed by the two methods of A.D. and LiMa.}
\label{f:ad-vs-lima}
\end{figure}

\begin{figure}[t]
\centering
\includegraphics[width=1\textwidth]{figures/experiments/ad_vs_li_ma_first_detections_testset_h_id_1.png}
\caption{Result on Test Set H: 5$\sigma$ detection times distributions for A.D. and LiMa over 419 GRB trials.}
\label{f:ad-vs-lima-first-detection}
\end{figure}


\begin{figure}[t]
\centering
\includegraphics[width=1\textwidth]{figures/experiments/ad_vs_li_cumulative_testset_h_id_1.png}
\caption{Result on Test Set H: Total number of 5$\sigma$ detections (out of 419 GRB trials) performed by the two methods of A.D. and LiMa.}
\label{f:ad-vs-lima}
\end{figure}


Fig~\ref{f:ad-vs-lima-first-detection} and Fig~\ref{f:ad-vs-lima} show the number of predictions at shorter time scales, limiting the duration of the observation. 
% Li&Ma è nel caso peggiore ovvero il bin temporale non sta mai a cavallo dell'onset
The A.D. method is capable of faster prediction on the very short-term temporal scale and the performance of the RNN model, are always better against Li\&Ma. 
 











\subsection{Science alert use case}
\label{s:Science-Alert-Results}
The science-alert use case further limits the time domain. When the observatory reacts to a science alert, its telescopes will change their pointing to observe a sky region in which the GRB event is already started. The time taken to change the pointing is considered fixed and equal to 30 seconds.  



\begin{figure}
    
    \includegraphics[width=1\textwidth]{figures/experiments/detection_plots/detection_plot_run0214_ID000101_testset_e.png}\hfill
    \\[\smallskipamount]
    
    \includegraphics[width=1\textwidth]{figures/experiments/detection_plots/detection_plot_run0214_ID000101_testset_e.png}\hfill
    \\[\smallskipamount]

    \includegraphics[width=1\textwidth]{figures/experiments/detection_plots/detection_plot_run0350_ID000189_testset_e.png}\hfill
    \\[\smallskipamount]

    \caption{Detection plots 1}\label{fig:detection-plots-1}
\end{figure}


\begin{figure}
    \includegraphics[width=1\textwidth]{figures/experiments/detection_plots/detection_plot_run0428_ID000157_testset_e.png}\hfill
    \\[\smallskipamount]

    \includegraphics[width=1\textwidth]{figures/experiments/detection_plots/detection_plot_run0032_ID000190_testset_e.png}\hfill
    \\[\smallskipamount]
    
    \includegraphics[width=1\textwidth]{figures/experiments/detection_plots/detection_plot_run0055_ID000128_testset_e.png}\hfill
    \\[\smallskipamount]
    \caption{Detection plots 2}\label{fig:detection-plots-2}
\end{figure}



\begin{figure}
    \includegraphics[width=1\textwidth]{figures/experiments/detection_plots/detection_plot_run0508_ID000316_testset_e.png}\hfill
    \\[\smallskipamount]

    \includegraphics[width=1\textwidth]{figures/experiments/detection_plots/detection_plot_run0642_ID000033_testset_e.png}\hfill
    \\[\smallskipamount]
    
    \includegraphics[width=1\textwidth]{figures/experiments/detection_plots/detection_plot_run0881_ID000086_testset_e.png}\hfill
    \\[\smallskipamount]
    \caption{Detection plots 2}\label{fig:detection-plots-2}
\end{figure}

\section{Summary}
\label{s:Experiments-Summary}

Summarize the key findings of the experiments you conducted.
