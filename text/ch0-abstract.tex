\textit{Context}. The Cherenkov Telescope Array (CTA) will be the next-generation ground-based observatory for observing the universe in the very-high-energy domain. It will bring advancements to gamma-ray astronomy by deploying over sixty highly sensitive Cherenkov telescopes, with a sensitivity improvement of one order of magnitude compared to current Imaging Atmospheric Cherenkov Telescopes (IACTs). The observatory will rely on a Science Alert Generation (SAG) system to analyze the real-time data from the telescopes and generate science alerts. The SAG system will play a crucial role in the search and follow-up of transients from external alerts, enabling multi-wavelength and multi-messenger collaborations. The observatory will operate arrays on sites in both hemispheres to provide full sky coverage. It will maximize the potential for the rarest phenomena, such as gamma-ray bursts (GRBs), which are the science case for this study.\\
\textit{Aims}. The thesis aims to investigate the usage of anomaly detection for real-time gamma-ray analysis. A deep learning-based technique has been developed to pursue this goal.\\
\textit{Results}. This study presents an anomaly detection method based on deep learning for detecting gamma-ray burst events in real-time. The performance of the proposed method is evaluated and compared against the Li\&Ma standard technique in two use cases of \textit{serendipitous discoveries} and \textit{follow-up observations}, using different exposure times. The method shows promising results in detecting GRBs and is flexible enough to allow real-time search for transient events on multiple time scales. The method does not assume background nor source models and doesn't require a minimum number of photon counts to perform analysis, making it well-suited for real-time analysis.\\
\textit{Conclusions}. Future improvements involve further tests relaxing some of the assumptions made in this study as well as post-trials correction of the detection significance. Moreover, the ability to detect other transient classes in different scenarios must be investigated for completeness. The system can be integrated within the SAG system of CTA and deployed on the onsite computing clusters. This would provide valuable insights into the method's performance in a real-world setting and be another valuable tool for discovering new transient events in real-time. Overall, this study makes a significant contribution to the field of astrophysics by demonstrating the effectiveness of deep learning-based anomaly detection techniques for real-time source detection in gamma-ray astronomy.